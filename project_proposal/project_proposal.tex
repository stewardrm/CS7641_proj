\documentclass[10pt,twocolumn,letterpaper]{article}

\usepackage{cvpr}
\usepackage{times}
\usepackage{epsfig}
\usepackage{graphicx}
\usepackage{amsmath}
\usepackage{amssymb}

% Include other packages here, before hyperref.

% If you comment hyperref and then uncomment it, you should delete
% egpaper.aux before re-running latex.  (Or just hit 'q' on the first latex
% run, let it finish, and you should be clear).
\usepackage[pagebackref=true,breaklinks=true,letterpaper=true,colorlinks,bookmarks=false]{hyperref}

\cvprfinalcopy % *** Uncomment this line for the final submission

\def\cvprPaperID{****} % *** Enter the CVPR Paper ID here
\def\httilde{\mbox{\tt\raisebox{-.5ex}{\symbol{126}}}}

% Pages are numbered in submission mode, and unnumbered in camera-ready
\ifcvprfinal\pagestyle{empty}\fi
\begin{document}

%%%%%%%%% TITLE
\title{Team Recommender}

\author{Greta Sharoyan\\
Georgia Institute of Technology\\
{\tt\small gsharoya@gatech.edu}
% For a paper whose authors are all at the same institution,
% omit the following lines up until the closing ``}''.
% Additional authors and addresses can be added with ``\and'',
% just like the second author.
% To save space, use either the email address or home page, not both
\and
Peng Chun\\
Georgia Institute of Technology\\
{\tt\small pchun9@gatech.edu}\\
\and
Amari Farnaz\\
Georgia Institute of Technology\\
{\tt\small farnaz\_amiri@gatech.edu}
\and
Robert Steward\\
Georgia Institute of Technology\\
{\tt\small rsteward7@gatech.edu}
}

\maketitle
%\thispagestyle{empty}

%%%%%%%%% ABSTRACT
\begin{abstract}
   Project summary (4-5+ sentences). Fill in your problem and background/motivation (why do you want to solve it? Why is it interesting?). This should provide some detail (don’t just say “I’ll be working on object detection”)
\end{abstract}

%%%%%%%%% BODY TEXT
\section{What we will do}
What you will do (Approach, 4-5+ sentences) - Be specific about what you will implement and what existing code you will use. Describe what you actually plan to implement or the experiments you might try, etc. Again, provide sufficient information describing exactly what you’ll do. One of the key things to note is that just downloading code and running it on a dataset is not sufficient for a description or a project! Some thorough implementation, analysis, theory, etc. have to be done for the project.

\section{Resources}

Resources  Related Work \& Papers (4-5 sentences). What is the state of art for this problem? Note that it is perfectly fine for this project to implement approaches that already exist. This part should show you’ve done some research about what approaches exist.


\section{Datasets}

Datasets (Provide a link to the dataset). This is crucial! Deep learning is data-driven, so what datasets you use is crucial. One of the key things is to make sure you don’t try to create and especially annotate your own data! Otherwise, the project will be taken over by this.


\end{document}
